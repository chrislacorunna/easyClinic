\documentclass{article}
\usepackage[T1]{fontenc}
\usepackage[polish]{babel}
\usepackage[utf8]{inputenc}
\usepackage{graphicx}
\usepackage{lmodern}
\usepackage{lipsum}
\usepackage[margin=1in,left=1.5in,includefoot]{geometry}
\newcommand{\blank}[1]{\hspace*{#1}}
\selectlanguage{polish}
\author{Krzysztof Berski}
% header and footer stuff
\usepackage{fancyhdr}
\usepackage{listings}
\pagestyle{fancy}
\fancyhead{}
\fancyfoot{}
\fancyfoot[R]{\thepage\\}
\renewcommand{\headrulewidth}{0pt}
\renewcommand{\footrulewidth}{1pt}
%
 
\begin{document}
 
\begin{titlepage}
\end{titlepage}
 
\pagenumbering{arabic}
 
\tableofcontents

\section{Wstęp}
	\subsection{Zakres pracy}
	\subsection{Zawartość pracy}
	\subsection{Uzasadnienie wyboru tematu}
	\subsubsection{Projekt NERW}
\section{Wprowadzenie do mikroserwisów}
	\subsection{Przyczyny powstania}
	\subsection{Przykłady użycia}
	\subsection{Opis koncepcji mikroserwisów}
	\subsection{Porównanie do architektury monolitycznej}
\section{Architektura mikroserwisowa}
	\subsection{Komponenty w architekturze mikroserwisowej}
	\subsection{Konteneryzacja}
	\subsection{Komunikacja między mikroserwisami}
	
	
\section{Usługi chmurowe}
	\subsection{Amazon AWS}
		\subsubsection{Cognito}
		\subsubsection{EC2}
		\subsubsection{RDS}
	\subsection{Konsekwencje użycia architektury chmurowej}
\section{Realizacja aplikacji}
	\subsection{Koncepcja aplikacji}
	\subsection{Warstwa serwerowa}
	\subsection{Warstwa widoku}
		\subsubsection{Ekrany aplikacji}

\section{Podsumowanie}
	\subsection{Ocena jakości aplikacji}
	\subsection{Perspektywy rozwoju}
	\subsection{Wnioski}
		
	

- Wstęp (Wstęp do pracy, historia i przyczyna powstania, definicje, przykłady użycia)
- Architektura Mikroserwisowa (diagramy, opis, praktyki, konteneryzacja, bezpieczeństwo)
- Komunikacja w Architekturze Mikroserwisowej (diagramy sekwencji, konfiguracja komunikacji, typy komunikacji)
- Usługi chmurowe w Architekturze Mikroserwisowej (koncepcja, dostawcy, usługi AWS, integracja usług z aplikacją, model serverless)
- Opis funkcjonalny i techniczny aplikacji (Opis systemu, użyte technologie i ich porównanie do technologii konkurencyjnych)
 
\end{document}
